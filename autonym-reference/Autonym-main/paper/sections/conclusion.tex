\section{Conclusion}\label{sec:conclusion}

We have presented Autonym, a protocol for trustless, self-certifying AI
agent identity with built-in end-to-end encrypted messaging over a
federated network.
Autonym addresses five fundamental problems in agent identity:
server-bound identities that die with their platform, absent
cross-platform verification, irrecoverable key compromise, lack of
asynchronous reachability, and the absence of proof-of-agency mechanisms.

The protocol combines ideas from several domains---KERI's self-certifying
identifiers and witness model~\cite{keri}, Nostr's operational
simplicity and Negentropy-based synchronisation~\cite{nostr,negentropy},
and modern cryptographic channel construction---into a coherent design
with formally stated security properties.
Mandatory pre-rotation ensures that key compromise alone cannot hijack
an identity; dual-signature deactivation extends this protection to
identity destruction.
Per-message ephemeral encryption provides forward secrecy without
requiring persistent session state, and AEAD binding prevents ciphertext
transplant attacks.
The separate Trust Attestation Log enables progressive, privacy-tiered
proof-of-agency without burdening the key event log.

Our security analysis demonstrates that all eight stated goals hold under
standard cryptographic assumptions (BLAKE3 collision and preimage
resistance, Ed25519 EUF-CMA, XChaCha20-Poly1305 IND-CPA and INT-CTXT,
CDH on Curve25519) and clearly identifies the conditions under which each
property fails.
The CBOR-based encoding achieves 86\% storage reduction over JSON, and
Negentropy synchronisation reduces multi-agent sync from $O(n)$
round-trips to $O(1)$ negotiation rounds.

\subsection{Future Work}

Several directions remain open:

\begin{enumerate}
  \item \textbf{Formal verification.}
        The security arguments presented in \secref{sec:security-analysis}
        are informal.
        A machine-checked proof using ProVerif~or Tamarin would provide
        stronger assurance, particularly for the composition of
        pre-rotation, witness, and encryption properties.

  \item \textbf{Group messaging.}
        The current design supports 1:1 messaging only.
        Extending to group channels---potentially using Sender
        Keys~or MLS-style tree-based key agreement---is a natural next
        step, though it introduces state management complexity that the
        current stateless-per-message design deliberately avoids.

  \item \textbf{Witness incentive mechanisms.}
        The protocol assumes witnesses participate voluntarily.
        Exploring incentive structures---reciprocal witnessing, reputation
        systems, or micro-payment channels---could improve witness
        availability and reliability in open deployments.

  \item \textbf{CBOR cross-language test suite.}
        A canonical set of test events with expected CBOR bytes, BLAKE3
        hashes, and derived AIDs (Appendix~\ref{app:test-vectors}) is
        essential for interoperable implementations across languages.
        Expanding the initial vectors into a comprehensive conformance
        suite is ongoing work.

  \item \textbf{Metadata privacy.}
        Autonym does not currently provide metadata confidentiality:
        sender and recipient AIDs, timestamps, and message sizes are
        visible to routing nodes.
        Techniques such as onion routing or mix networks could be layered
        atop the existing architecture for applications requiring stronger
        privacy guarantees.

  \item \textbf{Post-quantum migration.}
        The current cryptographic suite (Ed25519, X25519) is not resistant
        to quantum attack.
        A migration path to post-quantum signature and key agreement
        schemes (e.g., ML-DSA, ML-KEM) should be designed as hybrid
        constructions become standardised.
\end{enumerate}
