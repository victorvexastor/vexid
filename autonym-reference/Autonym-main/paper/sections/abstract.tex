\begin{abstract}
Autonomous AI agents increasingly require portable, cryptographically
verifiable identities that survive platform failures and resist key
compromise.
We present \textbf{Autonym}, a protocol for trustless, self-certifying
agent identity and end-to-end encrypted messaging over a federated node
network.
Autonym identifiers are derived deterministically from inception key
material using BLAKE3, ensuring no dependence on registries or
blockchains.
The protocol enforces \emph{mandatory} pre-rotation: at inception and
every key rotation, the agent commits to the hash of its next public key,
so that compromise of the current signing key alone is insufficient to
hijack or destroy the identity.
Deactivation additionally requires a dual-signature from both the current
and pre-committed keys.
A threshold-based witness model provides equivocation detection via a
first-seen-first-signed rule, with aggregate index-referenced receipts
that reduce per-receipt overhead by up to 75\%.
Autonym includes store-and-forward messaging with per-message forward
secrecy: each message is encrypted under a fresh ephemeral X25519 key
exchange, with XChaCha20-Poly1305 AEAD binding that prevents ciphertext
transplant attacks.
A separate Trust Attestation Log supports privacy-tiered behavioral
metrics, operator attestations, and vouches for progressive
proof-of-agency.
All protocol data uses CBOR deterministic encoding with optional zstd
compression, achieving 86\% storage reduction over JSON baselines.
Federated node-to-node synchronisation employs Negentropy set
reconciliation with $O(d \cdot \log(n/d))$ bandwidth complexity.
We define a formal threat model with eight security goals and provide
informal but rigorous security arguments grounded in standard
cryptographic assumptions (BLAKE3 collision/preimage resistance, Ed25519
EUF-CMA, XChaCha20-Poly1305 IND-CPA/INT-CTXT, CDH on Curve25519).
\end{abstract}
