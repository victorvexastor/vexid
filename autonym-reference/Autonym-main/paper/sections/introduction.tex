\section{Introduction}\label{sec:introduction}

The proliferation of autonomous AI agents---software entities that
interact with services, other agents, and humans on behalf of their
operators---has created an urgent need for portable, cryptographically
verifiable identity infrastructure.
Current approaches tie agent identity to specific platforms: if the
platform disappears, every agent's identity dies with it.
Cross-platform identity verification is impossible when each platform
maintains its own siloed registry.
Key compromise under most existing schemes results in permanent identity
loss, with no recovery path that does not involve trusting a central
authority.
Furthermore, two agents whose sessions do not overlap have no standard
mechanism for asynchronous, encrypted communication.

Existing decentralised identity protocols address subsets of these
problems.
KERI~\cite{keri,keri-ietf} introduced self-certifying identifiers with
key event logs, pre-rotation, and witnesses, but provides no built-in
messaging and uses a specialised encoding (CESR) that limits adoption.
Nostr~\cite{nostr} offers simple relay-based communication but lacks key
rotation entirely.
The AT Protocol~\cite{atproto} relies on a custodial PLC directory.
W3C DIDs~\cite{did-core} provide a broad framework, but security
properties vary dramatically across DID methods, and most methods still
depend on external registries.
DIDComm~\cite{didcomm} layers messaging atop DIDs but does not provide
per-message forward secrecy.
None of these protocols combine self-certifying identity, mandatory
pre-rotation, witness-based equivocation detection, end-to-end encrypted
messaging with forward secrecy, and agent-specific attestations in a
single coherent design.

\paragraph{Contributions.}
This paper presents \textbf{Autonym}, a protocol that addresses the full
spectrum of AI agent identity requirements.  Our contributions are:

\begin{enumerate}[label=\textbf{C\arabic*:},leftmargin=2.5em]
  \item \textbf{Self-certifying identity with mandatory pre-rotation and
        dual-signature deactivation.}
        Autonym identifiers (\aid{}s) are BLAKE3 digests of the inception
        event, requiring no external registry.
        Pre-rotation is unconditional: every event commits to the next
        public key hash.
        Deactivation requires signatures from both the current and
        pre-committed keys, preventing an attacker with only the current
        key from destroying the identity (\secref{sec:aid-derivation},
        \secref{sec:kel}).

  \item \textbf{Separate Trust Attestation Log with privacy-tiered
        behavioural metrics.}
        A hash-chained \tal{} stores operator attestations, behavioural
        observations (in three privacy tiers), and time-bounded vouches,
        independently prunable from the \kel{}
        (\secref{sec:tal}).

  \item \textbf{End-to-end messaging with per-message forward secrecy.}
        Each message is encrypted under a fresh ephemeral X25519 key
        exchange, with XChaCha20-Poly1305 AEAD binding that prevents
        ciphertext transplant attacks.
        The sender deletes the ephemeral private key immediately after
        encryption (\secref{sec:e2e-encryption}).

  \item \textbf{Negentropy-based federated synchronisation.}
        Node-to-node synchronisation uses Negentropy set
        reconciliation~\cite{negentropy}, achieving $O(d \cdot \log(n/d))$
        bandwidth for $d$ differences among $n$ total events
        (\secref{sec:sync}).

  \item \textbf{Compact deterministic encoding.}
        All protocol data uses CBOR deterministic encoding~\cite{rfc8949}
        with optional zstd compression~\cite{rfc8478}, achieving 86\%
        storage reduction over equivalent JSON representations
        (\secref{sec:performance}).

  \item \textbf{Formal threat model and security analysis.}
        We define eight security goals under a modified Dolev--Yao
        adversary model~\cite{dolev-yao} and provide informal but rigorous
        arguments for each, grounded in standard cryptographic assumptions
        (\secref{sec:threat-model}, \secref{sec:security-analysis}).
\end{enumerate}

\paragraph{Organisation.}
\secref{sec:related-work} surveys related work.
\secref{sec:preliminaries} introduces notation and cryptographic
preliminaries.
\secref{sec:threat-model} defines the threat model and security goals.
\secref{sec:protocol} describes the protocol in detail.
\secref{sec:security-analysis} analyses each security goal.
\secref{sec:performance} evaluates storage, bandwidth, and compression.
\secref{sec:comparison} compares Autonym to existing protocols.
\secref{sec:implementation} outlines the reference implementation.
\secref{sec:conclusion} concludes with future work.
Appendices provide test vectors, error codes, and CBOR schemas.
