\section{Related Work}\label{sec:related-work}

Autonym draws on a rich landscape of prior work spanning self-certifying identity, key management, decentralized messaging, secure channel construction, and the nascent field of agent-specific identity.
This section surveys each domain, identifies the contributions most relevant to Autonym's design, and highlights the gaps that motivate the protocol.

%% ─────────────────────────────────────────────────────────────────
\subsection{Self-Certifying Identity}\label{sec:rw-identity}

\paragraph{KERI.}
The Key Event Receipt Infrastructure~\cite{keri,keri-ietf} is Autonym's most direct ancestor.
KERI introduced Key Event Logs (\kel{}s)---append-only, hash-chained sequences of key lifecycle events---together with self-certifying identifiers derived from inception events, pre-rotation commitments, and a witness model for equivocation detection.
Autonym inherits all four concepts but departs from KERI in several ways.
First, KERI employs CESR (Composable Event Streaming Representation), a custom text-binary hybrid encoding; Autonym replaces this with deterministic CBOR~\cite{rfc8949}, which enjoys broad tooling support and an unambiguous canonical form (RFC~8949~\S4.2).
Second, KERI supports a large matrix of key types and multi-signature schemes, increasing implementation surface; Autonym restricts to Ed25519~\cite{ed25519-security} to minimize complexity.
Third, KERI is identity-only---it deliberately excludes messaging; Autonym integrates store-and-forward messaging with end-to-end encryption as a first-class protocol layer (\secref{sec:messaging}).
Finally, KERI introduces domain-specific terminology (TELs, ACDCs, OOBIs, SAIDs) that can impede adoption; Autonym favours conventional names (\kel{}, \tal{}, service endpoints).

\paragraph{W3C DIDs.}
The Decentralized Identifiers specification~\cite{did-core} provides a broad framework for decentralized identity through method-specific resolution.
DID Documents describe verification methods, service endpoints, and controller relationships.
However, the security properties of a DID are entirely method-dependent: \texttt{did:web} trusts DNS and TLS, \texttt{did:key} offers no rotation, and \texttt{did:ion} requires a Bitcoin full node.
Autonym provides DID interoperability by generating conformant DID Documents on the fly from \kel{}s (as \texttt{did:autonym}), but the \aid{} remains the canonical identifier, ensuring that security properties do not degrade through method-layer indirection.

\paragraph{DID:PLC (AT Protocol).}
The AT Protocol~\cite{atproto} powers Bluesky and introduces \texttt{did:plc}, a DID method backed by a centralized PLC (Public Ledger of Credentials) directory.
PLC supports key rotation via signed operations submitted to the directory, but the PDS (Personal Data Server) typically holds signing keys custodially on behalf of the user.
This trades self-sovereignty for usability: account recovery is possible through the PLC directory, but the directory becomes a single point of trust.
Autonym avoids custodial key management entirely---the agent always holds its own Ed25519 private key, and recovery relies on pre-rotation commitments rather than a trusted third party.

%% ─────────────────────────────────────────────────────────────────
\subsection{Key Management}\label{sec:rw-keys}

\paragraph{Pre-rotation.}
KERI KID-0005~\cite{keri-prerotation} introduced pre-rotation: at inception, the controller commits to the hash of the next public key ($\nkh = \hash{\npk}$).
A key compromise cannot produce a valid rotation because the attacker does not possess the pre-committed next key.
In KERI, pre-rotation is supported but optional; identifiers may be created without a next-key commitment.
Autonym makes pre-rotation \emph{mandatory}---every inception and rotation event MUST include the \fld{n} field---closing the gap where identifiers could exist without compromise recovery.

\paragraph{UCAN.}
Universal Capability Authorization Tokens~\cite{ucan} provide a capability-based authorization framework built on signed JWT chains.
UCANs enable delegation (Alice can grant Bob a subset of her capabilities) and are complementary to identity systems.
Autonym does not implement capability delegation at the protocol layer, but its \tal{} structure (particularly operator attestations and vouches) could carry UCAN-style delegation proofs as an extension.

\paragraph{ZCAPs.}
Authorization Capabilities for Linked Data~\cite{zcaps} take a similar approach to delegated authorization, using JSON-LD and Linked Data Proofs.
ZCAPs support capability chains, attenuation, and revocation.
Like UCANs, ZCAPs address a complementary problem (authorization) rather than identity lifecycle management.

%% ─────────────────────────────────────────────────────────────────
\subsection{Decentralized Messaging}\label{sec:rw-messaging}

\paragraph{DIDComm.}
DIDComm Messaging v2.0~\cite{didcomm} is the Decentralized Identity Foundation's messaging layer for DID-based communication.
DIDComm provides rich message routing, content types, and protocol negotiation.
However, DIDComm is complex: it requires DID resolution, envelope encryption via JWE or ECDH-ES, and does not prescribe per-message forward secrecy by default.
Autonym's messaging layer is deliberately simpler---a single envelope format with mandatory per-message ephemeral key exchange providing forward secrecy (\secref{sec:e2e-encryption}).

\paragraph{Matrix.}
The Matrix protocol~\cite{matrix-spec} provides federated, end-to-end encrypted messaging using the Olm and Megolm ratchets (derived from the Signal Double Ratchet).
Matrix servers attest user identity---if a homeserver is compromised, so is the identity mapping.
Matrix is designed for human group communication with rich room semantics; Autonym targets point-to-point agent messaging with self-certifying identity that survives server compromise.

\paragraph{Nostr.}
Nostr~\cite{nostr} takes a minimalist approach: identities are bare secp256k1 public keys, and messages (events) are JSON objects signed and published to relays.
Nostr's simplicity has enabled rapid adoption, and its NIP-77~\cite{nip77} negentropy-based set reconciliation provides efficient relay synchronization.
Autonym adopts Negentropy sync~\cite{negentropy} directly from this lineage.
However, Nostr lacks key rotation (NIP-41 remains a draft with no finalized specification), has no witness model, and provides no native end-to-end encryption with forward secrecy.
Agent identity on Nostr is indistinguishable from human identity.

\paragraph{AgentMail.}
AgentMail~\cite{agentmail} provides email-like infrastructure specifically for AI agents, with mailboxes, threading, and API-first design.
Identity is centralized---agents register with the AgentMail service.
Autonym provides similar store-and-forward semantics but with self-certifying, federated identity that does not depend on any single service provider.

%% ─────────────────────────────────────────────────────────────────
\subsection{Secure Channels}\label{sec:rw-channels}

\paragraph{Signal Double Ratchet.}
The Signal protocol~\cite{signal-protocol} combines the Double Ratchet algorithm with X3DH key agreement to provide forward secrecy and future secrecy (post-compromise security) through continuous key ratcheting.
This is the state of the art for persistent messaging sessions.
However, the Double Ratchet requires both parties to maintain synchronized session state; if state is lost, messages become undecryptable.
Autonym's per-message ephemeral approach (\secref{sec:e2e-encryption}) trades ratcheting's future secrecy for statelessness: each message is an independent key agreement using a fresh X25519 ephemeral keypair.
This design suits AI agents that may be stateless, run across multiple instances, or restart frequently.

\paragraph{Noise Framework.}
The Noise Protocol Framework~\cite{noise-framework} defines a family of Diffie-Hellman-based handshake patterns with formal security properties.
Noise handshakes power WireGuard and the Lightning Network.
Autonym's per-message encryption resembles a degenerate Noise~\texttt{N} pattern (one-way, no static sender authentication via DH---instead, authentication is provided by the Ed25519 signature on the envelope), optimized for asynchronous, one-shot message delivery.

\paragraph{X3DH.}
The Extended Triple Diffie-Hellman protocol~\cite{x3dh} enables asynchronous key agreement by pre-publishing signed pre-keys.
X3DH is designed as the handshake for the Signal Double Ratchet.
Autonym does not use X3DH because it does not maintain ratchet sessions; instead, each message performs a single X25519 exchange between a fresh ephemeral key and the recipient's public key derived from their Ed25519 identity key via RFC~8032~\cite{rfc8032}.

%% ─────────────────────────────────────────────────────────────────
\subsection{Agent Identity Systems}\label{sec:rw-agent-identity}

The emergence of autonomous AI agents in 2025--2026 has prompted several domain-specific identity proposals.

\paragraph{NIST.}
The NIST concept paper on autonomous agent identity~\cite{nist-agent-identity} articulates the government perspective: agents need machine-readable credentials, lifecycle management, and auditability.
The paper identifies requirements that align with Autonym's design---cryptographic binding between identity and keys, rotation support, and attestation of agent provenance---but does not prescribe a protocol.

\paragraph{Clawstr.}
Clawstr~\cite{clawstr} deploys AI agents on the Nostr relay network, enabling agent-to-agent communication via Nostr events.
Clawstr inherits Nostr's limitations: no key rotation, secp256k1 identity with no pre-rotation, and no formal agent attestation.
Agent identity is indistinguishable from human identity at the protocol level.

\paragraph{OWASP ANS.}
The OWASP Agent Name Service~\cite{owasp-ans} proposes a DNS-like discovery layer for AI agents, mapping human-readable names to agent metadata including capabilities and trust signals.
ANS focuses on discovery and does not provide cryptographic identity lifecycle (key rotation, pre-rotation, equivocation detection).
Autonym's alias system and \texttt{/.well-known/autonym} endpoint serve a similar discovery function, but identity is grounded in self-certifying \aid{}s rather than a naming service.

\paragraph{ERC-8004.}
ERC-8004~\cite{erc8004} defines an Ethereum-based agent identity standard, binding agent identity to on-chain state.
This provides immutability and global consensus at the cost of Ethereum dependency (gas fees, transaction latency, chain availability).
Autonym's architecture explicitly avoids blockchain as a runtime dependency, though it supports optional anchoring of \kel{} Merkle roots to a chain for additional assurance.

%% ─────────────────────────────────────────────────────────────────
\subsection{Gap Analysis}\label{sec:rw-gap}

No single existing protocol combines
(i)~self-certifying identifiers with
(ii)~mandatory pre-rotation,
(iii)~witness-based equivocation detection,
(iv)~built-in end-to-end encrypted messaging with per-message forward secrecy,
(v)~agent-specific attestations (operator claims, behavioural metrics, vouches), and
(vi)~compact deterministic encoding with set-reconciliation sync.
KERI provides~(i)--(iii) but lacks~(iv)--(vi).
Nostr provides relay-based messaging and Negentropy sync but lacks~(i)--(v).
DIDComm provides messaging for DID holders but lacks mandatory pre-rotation, witnesses, and agent attestations.
AT Protocol provides rotation via a custodial directory but sacrifices self-sovereignty.

Autonym is positioned as a synthesis:
KERI's identity rigor (mandatory pre-rotation, witness receipts, self-certifying \aid{}s) combined with Nostr's operational simplicity (anyone can run a node, Negentropy reconciliation), augmented with built-in messaging, agent-specific attestations in a dedicated \tal{}, and compact CBOR~\cite{rfc8949} encoding throughout.
\tabref{tab:comparison} in \secref{sec:comparison} provides a feature-level comparison.
