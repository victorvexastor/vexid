\section{Protocol Comparison}\label{sec:comparison}

Table~\ref{tab:comparison} compares Autonym against five representative protocols across eight features critical to AI agent identity infrastructure.
A checkmark (\checkmark) indicates native support, a dash (---) indicates absence, and a tilde ($\sim$) indicates partial or external support.
Numeric superscripts detail Autonym's implementation; alphabetic superscripts note caveats for other protocols.

\begin{table*}[t]
\centering
\caption{Feature comparison of identity and messaging protocols for AI agents.}
\label{tab:comparison}
\small
\begin{tabular}{@{}l c c c c c c@{}}
\toprule
\textbf{Feature} & \textbf{Autonym} & \textbf{KERI} & \textbf{Nostr} & \textbf{AT Proto} & \textbf{ActivityPub} & \textbf{DIDComm} \\
\midrule
Self-certifying IDs
  & \checkmark\textsuperscript{1} & \checkmark & \checkmark\textsuperscript{a} & ---\textsuperscript{b} & --- & $\sim$\textsuperscript{c} \\
Mandatory pre-rotation
  & \checkmark\textsuperscript{2} & $\sim$\textsuperscript{d} & --- & --- & --- & --- \\
Witness equivocation detection
  & \checkmark\textsuperscript{3} & \checkmark & --- & --- & --- & --- \\
Built-in messaging
  & \checkmark\textsuperscript{4} & --- & \checkmark & \checkmark\textsuperscript{e} & \checkmark & \checkmark \\
E2E encryption w/ forward secrecy
  & \checkmark\textsuperscript{5} & --- & $\sim$\textsuperscript{f} & --- & --- & $\sim$\textsuperscript{g} \\
Agent attestations
  & \checkmark\textsuperscript{6} & --- & --- & --- & --- & --- \\
Deterministic encoding
  & \checkmark\textsuperscript{7} & \checkmark\textsuperscript{h} & --- & --- & --- & --- \\
Set reconciliation sync
  & \checkmark\textsuperscript{8} & --- & \checkmark\textsuperscript{i} & --- & --- & --- \\
\bottomrule
\end{tabular}

\vspace{0.5em}
\raggedright\footnotesize
\textit{Autonym implementation details ---}\\
\textsuperscript{1} BLAKE3-based AID derived from the inception event's canonical CBOR encoding (\secref{sec:aid-derivation}); 32-byte Ed25519 keys.\\
\textsuperscript{2} Required at inception and every rotation; deactivation requires dual signatures from current and pre-committed next key (\secref{sec:kel}).\\
\textsuperscript{3} Threshold-based with aggregate receipts; witnesses are index-referenced in the event's $\fld{w}$ array, reducing per-receipt overhead by 53--75\% (\secref{sec:witnesses}).\\
\textsuperscript{4} Store-and-forward via home nodes; typed (MIME), threaded, 256\,KiB body limit; local and federated delivery (\secref{sec:messaging}).\\
\textsuperscript{5} Per-message ephemeral X25519 key exchange, HKDF-BLAKE3 derivation, XChaCha20-Poly1305 AEAD with conversation-bound associated data (\secref{sec:e2e-encryption}).\\
\textsuperscript{6} Separate TAL with dual-signed operator attestations, privacy-tiered behavioural metrics (Tier~1 boolean, Tier~2 bucketed, Tier~3 never published), and confidence-weighted vouches (\secref{sec:tal}).\\
\textsuperscript{7} CBOR deterministic encoding (RFC~8949~\S4.2)~\cite{rfc8949} with optional zstd compression; inception event $\approx$340\,B CBOR, $\approx$85\,B compressed (86\% reduction).\\
\textsuperscript{8} Negentropy-based set reconciliation adopted from Nostr NIP-77~\cite{negentropy,nip77}; $O(d \cdot \log(n/d))$ bandwidth (\secref{sec:sync}).\\[0.3em]
\textit{Other protocol caveats ---}\\
\textsuperscript{a} Nostr identities are bare secp256k1 public keys---self-certifying but without lifecycle management (NIP-41 key rotation remains draft).\\
\textsuperscript{b} AT Protocol uses \texttt{did:plc} with a custodial PLC directory; identifiers are not derived solely from key material.\\
\textsuperscript{c} DIDComm operates atop DIDs, whose self-certifying properties depend on the DID method used (\texttt{did:key} is self-certifying; \texttt{did:web} is not).\\
\textsuperscript{d} KERI supports pre-rotation but does not mandate it; identifiers may be created without a next-key commitment.\\
\textsuperscript{e} AT Protocol messaging is mediated through PDS infrastructure via Lexicon schemas (e.g., Bluesky DMs), not a standalone messaging layer.\\
\textsuperscript{f} Nostr NIP-44 provides encrypted direct messages using XChaCha20-Poly1305 but without per-message ephemeral key exchange for forward secrecy.\\
\textsuperscript{g} DIDComm v2 supports encryption via ECDH-ES+A256KW but does not mandate per-message ephemeral key exchange for forward secrecy.\\
\textsuperscript{h} KERI uses CESR (Composable Event Streaming Representation), a custom deterministic text-binary hybrid encoding optimised for streaming.\\
\textsuperscript{i} Nostr NIP-77 uses Negentropy-based set reconciliation~\cite{negentropy,nip77}.
\end{table*}

\subsection{Feature Discussion}\label{sec:comparison-discussion}

\paragraph{Self-certifying identifiers.}
A self-certifying identifier is derived deterministically from the holder's key material, eliminating dependence on external registries or certificate authorities.
In Autonym, the \aid{} is the base58-encoded, 256-bit BLAKE3 hash of the inception event's canonical CBOR encoding (\algref{alg:aid-derivation}).
The derivation uses a two-pass process: the first pass computes the \aid{} over the event content with self-referential fields zeroed; the second pass computes the event digest~$\fld{d}$ with the \aid{} populated (\defref{def:aid-valid}).
This property is essential for AI agents, which may operate across organisational boundaries and must prove identity without relying on any single platform's database.
Nostr's bare secp256k1 keys are self-certifying in the same sense but lack lifecycle management---there is no standard mechanism for key rotation or recovery.
AT Protocol's \texttt{did:plc} identifiers are bound to a centralised PLC directory, sacrificing the self-certifying property.

\paragraph{Mandatory pre-rotation.}
Pre-rotation commits to the hash of the next public key ($\nkh = \hash{\npk}$) at each event, ensuring that a compromised current key cannot produce a valid rotation (\secref{sec:kel}).
Every Autonym inception and rotation event \textbf{must} include the \fld{n} field; there is no opt-out.
Rotation is verified unconditionally: $\hash{\pk'} = \nkh_{\text{prev}}$, and is signed by the \emph{previous} key to prove authorisation (\algref{alg:kel-verify}, line~16).
Deactivation further requires dual signatures---both the current key ($\fld{sig}$) and the pre-committed next key ($\fld{ns}$)---preventing a key-compromise attacker from destroying the identity (\algref{alg:kel-verify}, lines~21--22).
KERI introduced pre-rotation~\cite{keri-prerotation} but does not require it, leaving some KERI identifiers vulnerable to key compromise without recovery.
For AI agents, mandatory pre-rotation is particularly important: agents may run on shared infrastructure where key material is at higher risk of exfiltration, and automated key rotation (without human intervention) requires a deterministic, pre-committed recovery path.

\paragraph{Witness-based equivocation detection.}
Witnesses co-sign key events and enforce a first-seen-first-signed rule (\defref{def:fsfs}): if a controller publishes conflicting events at the same sequence number, the fork is detectable through inconsistent witness receipts (\algref{alg:witness}).
Autonym uses index-based aggregate receipts---each receipt is a pair $(j, \sigma_j)$ referencing the witness's index in the event's $\fld{w}$ array---reducing per-receipt overhead by 53--75\% compared to full \aid{} references.
The witness threshold $\wt$ must satisfy $1 \le \wt \le |\wset|$; the recommended value is $\wt \ge \lceil |\wset|/2 \rceil + 1$.
When the witness set changes during rotation, receipts from $\wt$ of the \emph{outgoing} witness set are required, preventing silent witness replacement.
Among the compared protocols, only KERI and Autonym provide witness-based equivocation detection.
For AI agents interacting across trust boundaries, equivocation detection is critical: an agent that appears to hold different keys to different counterparts could sign contradictory commitments, making witness-based consistency a prerequisite for multi-party agent workflows.

\paragraph{Built-in messaging.}
Autonym integrates store-and-forward messaging directly into the protocol (\secref{sec:messaging}): agents send and receive CBOR-encoded, Ed25519-signed messages through their home nodes with at-least-once delivery semantics.
Messages are limited to 256\,KiB and carry a unique 16-byte identifier for idempotent duplicate handling.
Both local delivery (sender and recipient on the same node) and federated delivery (cross-node via \texttt{POST /federate/message}) are supported, with discovery via routing hints, cached \kel{}s, or peer queries.
Push delivery is available via WebSocket (\texttt{/messages/stream}) or webhooks.
KERI deliberately excludes messaging; Nostr and ActivityPub provide messaging but without self-certifying identity lifecycle management; DIDComm provides messaging but requires separate DID resolution infrastructure.
Coupling identity and messaging in a single protocol reduces the integration surface for AI agents: there is no gap between ``who is this agent?'' and ``how do I reach it?''---both answers derive from the same \kel.

\paragraph{End-to-end encryption with forward secrecy.}
Autonym provides per-message forward secrecy through ephemeral X25519 key exchange (\algref{alg:e2e}): each message generates a fresh X25519 keypair (32-byte public key), performs a Diffie-Hellman exchange with the recipient's X25519 public key (derived from their Ed25519 identity key via the RFC~8032~\cite{rfc8032} birational mapping), derives a 32-byte symmetric key via $\kdf{\mathit{ss},\; \mathit{salt},\; \texttt{"autonym-e2e-v1"}}$, and encrypts with XChaCha20-Poly1305~\cite{xchacha} using a random 24-byte nonce.
The associated data binds ciphertext to conversation metadata (sender \aid, recipient \aid, timestamp, ephemeral public key), preventing ciphertext transplant attacks~\cite{rfc8439}.
The ephemeral private key is deleted immediately after encryption, ensuring forward secrecy: compromising the sender's long-term key does not reveal past messages.
This stateless, per-message approach suits AI agents that may restart, scale horizontally, or lack persistent session storage---unlike the Signal Double Ratchet~\cite{signal-protocol}, which requires synchronised session state between parties.
Nostr's NIP-44 and DIDComm's ECDH-ES provide encryption but do not mandate per-message ephemeral exchange.

\paragraph{Agent attestations.}
Autonym's Trust Attestation Log (\tal{}, \secref{sec:tal}) supports three attestation types, each hash-chained and bound to a specific \kel{} state via $\fld{kel\_s}$ and $\fld{kel\_d}$:
\emph{operator attestations} bind an agent to a named operator with dual signatures (agent key and independent operator key $\pk_{\text{op}}$);
\emph{behavioural attestations} report observed agent behaviour using three privacy tiers (\defref{def:privacy-tiers})---Tier~1 (boolean flags), Tier~2 (bucketed ranges), and Tier~3 (raw metrics, never published);
\emph{vouches} provide confidence-weighted endorsements ($\fld{confidence} \in [0,1]$) with optional expiry.
No other compared protocol provides agent-specific attestation structures.

\paragraph{Deterministic encoding.}
Deterministic serialisation---where a given logical object has exactly one valid byte representation---eliminates canonicalisation attacks and simplifies signature verification.
Autonym uses CBOR deterministic encoding~\cite{rfc8949} (RFC~8949~\S4.2) for all protocol data: key events, attestations, messages, and receipts.
An inception event is approximately 340~bytes in CBOR versus ${\sim}650$~bytes in equivalent JSON; with zstd compression using a trained dictionary~\cite{rfc8478}, this drops to ${\sim}50$--85~bytes.
KERI uses CESR, a custom text-binary hybrid encoding with broader type support but less ecosystem tooling.
Protocols based on JSON (Nostr, ActivityPub, DIDComm) must canonicalise before signing, introducing potential ambiguity.
For AI agents that must verify signatures across diverse runtime environments (cloud, edge, WASM), deterministic encoding removes an entire class of interoperability bugs: the same event always produces the same bytes, regardless of the serialisation library used.

\paragraph{Set reconciliation sync.}
Negentropy-based set reconciliation~\cite{negentropy,nip77} enables two nodes to discover their differences in $O(d \cdot \log(n/d))$ bandwidth, where $d$ is the number of missing items and $n$ is the total set size (\secref{sec:sync}).
Missing events are transferred as CBOR Sequences~\cite{rfc8742}, optionally zstd-compressed.
This reduces multi-agent sync from $O(\text{agents})$ individual round-trips to $O(1)$ negotiation rounds.
Per-AID delta sync (\texttt{?from\_seq=N}) remains available for single-agent resolution.
Both Autonym and Nostr (NIP-77) adopt Negentropy; other protocols rely on per-item polling or full replication.
For AI agents deployed across federated infrastructure, efficient sync is essential: an agent that migrates between nodes or recovers from downtime must reconcile its event history without transferring the entire log.

\paragraph{Quantitative profile.}
Autonym's choice of CBOR deterministic encoding with zstd compression yields compact wire representations suitable for resource-constrained deployments.
An inception event is $\approx$340~bytes in CBOR (versus $\approx$650~bytes in equivalent JSON), compressing to $\approx$50--85~bytes with a trained zstd dictionary---an 86\% reduction overall.
Ed25519 public keys are 32~bytes; signatures are 64~bytes; BLAKE3 digests are 32~bytes.
Per-message E2E encryption adds 56~bytes of overhead: one ephemeral X25519 public key (32~bytes) and one XChaCha20 nonce (24~bytes).
At scale, a node hosting one million agents with five events each stores $\approx$5~GB compressed, compared to $\approx$35~GB in uncompressed JSON---enabling commodity hardware deployment without dedicated storage infrastructure.
